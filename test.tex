\documentclass[12pt,a4paper]{article}
\author{}

\usepackage[brazil]{babel}
\usepackage[utf8]{inputenc}
\usepackage[T1]{fontenc}
\usepackage{indentfirst}
\title{Waterfall}

\begin{document}
\maketitle
\vfill
\begin{center}
Gabriela Frizoni Carneiro - 
\\
Alexandre Cabral Bedeschi - 201569001A
\end{center}
\newpage

\section{Hardware e Software}

\subsection{Hardware}
\begin{itemize}
    \item Arduino
    \item Jumpers
    \item Protoboard
    \item Módulo Bluetooth HC-05
    \item Mini bomba d'água
    \item Ponte h
    \item Sensores de nível de fluido
    \item Fonte 12v
\end{itemize}

\subsection{Softwares}

\begin{itemize}
    \item Visual Studio Code + PlataformIO
    \item Flutter SDK
    \item FreeRTOS
\end{itemize}

\subsection{Links}
Códigos: https://github.com/o-mago/waterfall

App layout: https://www.figma.com/file/ed6Xty4baqywVBbnS5CsNm/Waterfall?node-id=0%3A2

\newpage
\section{Funcionamento}
\subsection{Arduino}
O reservatório de água em questão possui dois sensores de nível de fluido 
(um para nível baixo e outro para nível alto). O arduino recebe informações 
desses sensores nas suas portas digitais e trata esses sinais através de 
interrupções.

Supondo o reservatório vazio inicialmente, o sensor de "nível baixo" estará 
com valor LOW, o arduino ativará
a bomba através da task "TaskAgua". A bomba funcionará até o momento em que 
o sensor de "nível alto" for acionado, quando isso ocorrer, a bomba cessará 
seu funcionamento.

Nesse meio tempo, outra task estará em execução ("TaskBluetooth"). Ela é 
responsável pela comunicação Aplicação Mobile -> Arduino. Assim que alguma 
mensagem for recebida pelo módulo e alocada em um buffer no arduino, a mensagem
será lida e são quatro as instruções possíveis:
\begin{itemize}
\item Ligar o sistema
\item Desligar o sistema (desliga a bomba e trava a "TaskAgua")
\item Aumentar a potência da bomba (controle de tensão pela ponte h)
\item Diminuir a potência da bomba (controle de tensão pela ponte h)
\end{itemize}

A comunicação Arduino -> Aplicação Mobile é feita no momento das interrupções, 
enviando a informação referente ao nível da água para gerar a visibilidade no mesmo

\subsection{Aplicação Mobile}
A aplicação apresenta 4 telas:
\begin{enumerate}
    \item Tela inicial com logo
    \item Apresenta um botão para ligar o bluetooth e iniciar a procura de dispositivos
    \item Mostra os dispositivos bluetooth próximos (indicando a potência do sinal em dB), permitindo parear e conectar
    \item Tela principal. Mostra o nível do fluido, permite ligar e desligar o sistema de controle de nível de água, controlar a potência da bomba e desconectar do dispositivo
\end{enumerate}

\newpage
\section{Arduino}
\subsection{Esquemático}
\subsection{O sistema montado}

\section{Aplicação Mobile}
\subsection{SplashScreen}
\subsection{Liga o bluetooth (caso desligado)}
\subsection{Seleciona o dispositivo a se conectar}
\subsection{Tela principal}
\end{document}